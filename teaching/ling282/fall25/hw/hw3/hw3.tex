\documentclass[11pt]{article}
\usepackage{hyperref}
\usepackage[margin=0.75in]{geometry}
\usepackage{amsmath}
\usepackage{enumitem}

\newcommand{\bos}{\textless s\textgreater\:}
\newcommand{\eos}{\textless /s\textgreater\:}
\newcommand{\pad}{PAD\:}

\begin{document}

\title{Ling 282/482 hw3}
\date{\vspace{-0.2in}Due 11PM on October 29, 2025}
\maketitle

\noindent In this assignment, you will work with recurrent neural networks (RNNs) for language modeling. You will answer written questions about RNN architectures (including LSTMs), padding and masking for variable-length sequences, and perplexity-based evaluation. You will also explore a PyTorch implementation of character-level RNN language models, and train models with different hyperparameters to understand their effects on generated text.

\noindent More specifically, you will:
\begin{itemize}
    \item Review details of RNN and LSTM internals
    \item Analyze masking and padding for variable-length sequences
    \item Compute perplexity for language model evaluation
    \item Understand the PyTorch implementation of RNN language models
    \item Train character-level language models and experiment with hyperparameters
    \item Analyze generated text and model behavior
\end{itemize}

\section*{Submission Instructions}
\noindent Please answer the following questions and submit your answers \textbf{as a PDF on Blackboard}. Your final submission will be the one that is graded unless you specify otherwise.


\section{RNN Language Models [31 pts]}

\noindent {\bf Q1: Understanding RNNs [4 pts]}
\begin{enumerate}[label=\alph*., itemsep=2em]
  \item What is the main limitation of feed-forward neural networks that is overcome by recurrent networks, and how do recurrent networks achieve this? [2 pts]
  \item The Vanilla RNN equation has the form $h_t = f(h_{t-1}, x_t)$.  What extra `ingredient' does the LSTM add to this general form?  What problem is the LSTM designed to solve? [2 pts]
\end{enumerate}

\vspace{2em}
\noindent {\bf Q2: LSTM Update [11 pts]}  One of the ``central'' equations in the LSTM computation is the following:
\[ c_t = f_t \odot c_{t-1} + i_t \odot \hat{c}_t \]
This equation performs an essential update of one part of the LSTM.  Please answer:
\begin{enumerate}[label=\alph*., itemsep=2em]
  \item What is $c_t$? [1 pt]
  \item What is the range of $f_t$ and what is its purpose? [2 pts]
  \item What is the range of $i_t$ and what is its purpose? [2 pts]
  \item What is $\hat{c}_t$? [1 pt]
  \item In your own words, describe how this equation implements the central ``update'' inside of an LSTM (2-3 sentences). [5 pts]
\end{enumerate}

\vspace{2em}
\noindent {\bf Q3: Counting parameters [6 pts]}  Let $d_e$ be the dimension of token embeddings and $d_h$ the hidden state size.  Focusing on just the recurrent cell (and so ignoring the embedding and output layers):
\begin{enumerate}[label=\alph*., itemsep=2em]
  \item How many parameters are there in a Vanilla RNN cell? [2 pts]
  \item How many parameters are there in an LSTM cell? [4 pts]
\end{enumerate}

\vspace{2em}
Note: for this problem, you can assume that the RNN cell is at the `bottom' of a possibly-deep RNN, so the inputs to the cell are token embeddings, not earlier layers' hidden states.

\vspace{2em}
\noindent {\bf Q4: Understanding Masking [10 pts]}  Suppose that we want to train a (word-level) language model on the following two sentences:

\begin{center}
  \bos the cat sits \eos \\
  \bos the model reads the sentence \eos
\end{center}

As we saw in the slides, padding is necessary to make these sentences have the same length so that they can be batched together, as:

\begin{center}
  \bos the cat sits \eos \pad \pad \\
  \bos the model reads the sentence \eos
\end{center}

Please answer the following questions about these sequences:
\begin{enumerate}[label=\alph*., itemsep=2em]
  \item In a recurrent language model, what would the input batch be? What would the target labels be? Hint: think about what the \textit{input} and \textit{output} tokens are at each time-step. [2 pts]
  \item Recurrent language models use a \emph{mask} of ones and zeros to `eliminate' the loss for \pad tokens. What would the mask be for this batch? Your answer should be a matrix. [2 pts]
  \item Suppose that we have the following per-token losses:
    \[\begin{bmatrix}
      0.1 & 0.3 & 0.2 & 0.4 & 0.7 & 0.5 \\
      0.2 & 0.6 & 0.1 & 0.8 & 0.9 & 0.4
    \end{bmatrix}\]
    What is the corresponding \emph{masked} loss matrix? [2 pts]
  \item Why is it important to mask losses in this way?  What might a model learn to do if the loss is not masked? Answer in a few sentences. [4 pts]
\end{enumerate}

\section{Language Model Evaluation [16 pts]}

\noindent {\bf Q1: Evaluating Language Models [16 pts]} Given a corpus $W = w_1 w_2 \dots w_N$ (with $N$ as the number of tokens in the corpus), a common (intrinsic) evaluation metric for language models is \emph{perplexity}, defined as
\[ PP(W) = P(w_1 \dots w_N)^{-\frac{1}{N}} \]
This can be thought of as the inverse probability that the model assigns to the corpus, normalized by the size of the corpus.

\begin{enumerate}[label=\alph*., itemsep=2em]
  \item Is a lower or higher perplexity better? Why? [2 pts]
  \item For a recurrent language model, write an expression for $P(w_1 \dots w_N)$ using the chain rule of probability.  How is this different from the expression for a feed-forward language model? [4 pts]
  \item Show that
  \[ PP(W) = e^{-\frac{1}{N} \sum_{i=1}^N \log P(w_i \mid w_{<i})} \]
  where $w_{<i} = w_1 w_2 \dots w_{i-1}$ and $\log$ is the natural (base $e$) logarithm. [4 pts]
  \item What is another name for the exponent $-\frac{1}{N} \sum_{i=1}^N \log P(w_i \mid w_{<i})$ in the above expression? Hint: it appears in training as well. [2 pts]
  \item Suppose that the same text corpus were tokenized with two different vocabularies of different sizes (perhaps, e.g., one replaces infrequent tokens with an UNK token) and two language models were trained on the resulting tokenized text.  All else being equal, would you expect perplexity to be lower or higher for the model with a smaller vocabulary?  What consequences does this have for comparing different language models? [4 pts]
\end{enumerate}

\vspace{2em}
\section{Understanding the Implementation [26 pts]}

\noindent \textbf{Accessing the code:} You will receive an invitation to the Github repository for this assignment. As with hw2, you should fork the repository, clone your fork to BlueHive (or your local machine), and work from there. The code for this assignment can be found in the \texttt{decoding/} directory, which implements character-level language models. The main training script is \texttt{run.py}, which uses functions from \texttt{data.py} and model classes from \texttt{models.py}. Please answer the following questions about the implementation:

\vspace{2em}
\noindent {\bf Q1: Understanding LM Forward Passes [8 pts]}
\begin{enumerate}[label=\alph*., itemsep=2em]
  \item In \texttt{models.py}, \texttt{FeedforwardLanguageModel.forward}: in your own words, what are lines 62-68 doing? What would go wrong if we didn't include this block? Go beyond the information included in the comments. [3 pts]
  \item In the same function, line 77 is described as a ``flattening'' operation, implemented with PyTorch's \texttt{reshape} method. What is this ``flattening'' operation in the terms of the FFLM described in the slides and the textbook? What is its purpose? (Looking up the \texttt{reshape} method may help). [2 pts]
  \item In \texttt{RNNLanguageModel.forward}, lines 150-160, we use the functions \texttt{pack\_padded\_sequence} and \texttt{pad\_packed\_sequence}. Look up the PyTorch documentation for these functions. In general terms, what is their purpose? Again, go beyond the information provided in the comments. [3 pts]
\end{enumerate}

\vspace{2em}
\noindent {\bf Q2: Understanding masking [4 pts]} Look at the \texttt{mask\_loss} function in \texttt{run.py} (lines 33-61). Explain in 2-3 sentences why we divide by \texttt{mask.sum()} rather than by the total number of elements in the loss tensor. What would happen to the gradient magnitudes if we used the wrong denominator?

\vspace{2em}
\noindent {\bf Q3: Understanding generation [8 pts]}
\begin{enumerate}[label=\alph*., itemsep=2em]
  \item In 3-4 sentences of plain English, describe how the loop in the \texttt{generate} function in \texttt{run.py} works (lines 94-113). [4 pts]
  \item In line 92, we use the \texttt{torch.full} method. Look up the documentation for this, and briefly explain how it works. We used it for something similar in lines 64-66 of \texttt{models.py}. Briefly summarize the connection between the two calls to this same function. [2 pts]
  \item Why do we use \texttt{torch.no\_grad()} during generation (\texttt{run.py}, line 94)? What advantage does it provide? [2 pts]
\end{enumerate}

\vspace{2em}
\noindent {\bf Q4: GPU Training [6 pts]}
\begin{enumerate}[label=\alph*., itemsep=2em]
  \item The code uses \texttt{torch.device} and the \texttt{.to(device)} pattern throughout. Find where the device is determined and where it's used. Why is it important that the model, input data, and target labels all be on the same device? Why does the code determine the device flexibly at runtime rather than hardcoding whether to use GPU or CPU? [3 pts]
  \item Research and briefly explain (3-4 sentences): What makes GPUs faster than CPUs for neural network training? What kinds of operations benefit most from GPU acceleration? What kind of operations would not especially benefit? [3 pts]
\end{enumerate}

\vspace{2em}
\section{Running the Language Model [27 pts]}

\textbf{NOTE: Dedicated GPU resources for this assignment will be available on BlueHive within a few days of this homework's release. An updated version of this assignment with access instructions will be posted on Blackboard—watch for the announcement. You can complete Sections 1-3 now, but please wait for the updated instructions before starting Section 4. The code is runnable on laptops without GPU, but training will be significantly slower and may cause your processor to heat up.}

\texttt{run.py} contains a basic training loop for character-level language modeling. It will record the training and dev loss (and perplexity) at each epoch, and save the best model according to dev loss.  Periodically (as specified by a command-line flag), it also outputs generated text from the best model. \textbf{Note:} the reference implementation takes about 1 minute per epoch with default hyper-parameters. \textbf{Make sure you plan enough time to complete all three training runs in this part of the homework} (each will be at least 20 epochs).

\vspace{2em}
\noindent {\bf Q1: Default parameters [7 pts]} Execute \texttt{run.py} with its default arguments.  Paste below the texts that are generated every 4 epochs, as well as the epoch with the best dev loss and the dev perplexity from that epoch. In 3-4 sentences, describe any trends that you see in the generated text as training progresses. (Note that generated text will not necessarily be completely coherent: recall that this is a \emph{character-level} language model).

\vspace{2em}
\noindent {\bf Q2: Modify hyper-parameters [10 pts]} Re-run the training loop, modifying some combination of the following hyper-parameters, which are specified by command-line flags:
\begin{itemize}
  \item Model type (default: RNN. Options include LSTM and feedforward models)
  \item Hidden layer size
  \item Embedding size
  \item Batch size
  \item Learning rate
  \item Number of epochs (in particular: making it larger)
  \item Softmax temperature
  \item $L_2$ regularization coefficient
  \item Dropout (probability with which neurons are dropped during training)
\end{itemize}
Include your model's generated texts here. In 3-5 sentences, state exactly what hyper-parameter change(s) you made, and what effects (if any) you see in terms of the dev set perplexity and text that the model generated.

\vspace{2em}
\noindent {\bf Q3: Modify hyper-parameters again [10 pts]} Based on your results from Q1-Q2, pick another \textit{different} set of hyper-parameters to try. Before you re-run training, write down your prediction/hypothesis for what will happen. Re-run the training loop one more time, with the new hyper-parameters. In 3-4 sentences, state what hyper-parameter change(s) you made. Was the prediction you made supported? Report any trends and why you think your prediction was or was not born out.

\end{document}